\documentclass[11pt,letterpaper]{article}
\usepackage[utf8]{inputenc}
\begin{document}
	\section*{Key Points}
	\begin{itemize}
		\item Cardiovascular disease (CVD) has been the leading cause of death both in the US and worldwide, yet there exists a major research and funding disparity for CVD relative to cancer, especially with respect to computational work.
		\item A vast amount of past and present bioinformatics and computational biology research has been concentrated in precision oncology, opening up opportunities to apply and improve upon similar methodologies for CVD investigation, while keeping in mind some of the unique challenges inherent to cardioinformatics.
		\item The unprecedented growth of biomedical data and knowledge bases in the last several years is a clear indicator that informatics is becoming an integral component of basic and translational research.
		\item The complexity of CVDs calls for pushing beyond the traditional topics (e.g., single nucleotide polymorphisms) and methodologies (e.g., genome-wide association studies), pressing for more innovative and integrative applications of computational methods on more diverse sets of CVD data.
		\item We identified key challenges that, when addressed, would benefit CVD research and workflows immensely: data sharing and security, analytical data integration, and augmented intelligence for advanced data analysis at scale.  Although these challenges are common to many fields of data science, they are particularly pressing for researchers in cardioinformatics given the current state-of-the-art in CVD storage, computing, and infrastructure.
	\end{itemize}
	
\end{document}