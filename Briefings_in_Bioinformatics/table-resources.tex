\documentclass[12pt,letter]{article}
\usepackage{amsmath}
\usepackage{amsfonts}
\usepackage{amssymb}
\usepackage{graphicx}
\usepackage{lscape}
\begin{document}


\begin{landscape}
	\begin{table}[]
		\caption{List of available resources for cardioinformatics research}
	\label{tab:resources}
%		\resizebox{\textwidth}{!}{%
			\begin{tabular}{lp{8cm}ll}
			\hline
			\textbf{Type}      & \textbf{Name}                                   & \textbf{URL}                              & \textbf{Ref}                    \\ \hline
			Knowledge portal   & Cardiovascular Disease Knowledge Portal         & http://www.broadcvdi.org/                 & \cite{Broad:2018:Cardiovascular}      \\ \hline
			Knowledge portal   & Cerebrovascular Disease Knowledge Portal        & http://www.cerebrovascularportal.org/     & \cite{Crawford:2018:Cerebrovascular}  \\ \hline
			Knowledge portal   & HeartBioPortal                                  & https://heartbioportal.com/               & \cite{Khomtchouk:2019:HeartBioPortal} \\ \hline
			Analytics platform & AHA Precision Medicine Platform                 & https://precision.heart.org/              & \cite{Kass-Hout:2018:American}        \\ \hline
			Analytics platform & DataSTAGE                                       & https://datastage.io/                     & In planning                           \\ \hline
			Database           & HGDB (Heart Gene Database)                      & http://www.hgdb.ir/                       & \cite{Noorabad-Ghahroodi:2017:HGDB}   \\ \hline
			Database           & In-Cardiome (Integrated Cardiome Database)      & http://www.tri-incardiome.org/            & \cite{Sharma:2017:InCardiome}         \\ \hline
			Database           & Cardio/Vascular Disease Database                & http://www.padb.org/cvd/                  & \cite{Fernandes:2018:CVDdb}           \\ \hline
			Database           & CardioGenBase                                   & Discontinued                              & \cite{V:2015:CardioGenBase}           \\ \hline
			Review paper       & \multicolumn{2}{l|}{Cloud computing for genomic data analysis and collaboration}            & \cite{Langmead:2018:Cloud}            \\ \hline
			Review paper       & \multicolumn{2}{l|}{Human genotype–phenotype databases: aims, challenges and opportunities} & \cite{Brookes:2015:Human}             \\ \hline
			Review paper       & \multicolumn{2}{l|}{Methods of integrating data to uncover genotype–phenotype interactions} & \cite{Ritchie:2015:Methods}           \\ \hline
		\end{tabular}	\\ \newline
	\footnotesize{Resource types are categorized as following:\\
\textit{Database}: integrated datasets, harmonized and built into a single, searchable database. Query results are usually presented as table of text and hyperlinks.
\\
\textit{Knowledge portal}: integrated datasets, harmonized and built into a single, searchable database. Query results are usually visualized with charts tailored for the biological data and insights.
\\
\textit{Analytic platform}: computing system usually come with access to diverse datasets, allowing users to perform various analysis on the hosted data.	
}
	\end{table}
\end{landscape}

\bibliographystyle{plain}
\bibliography{cardio}
\end{document}